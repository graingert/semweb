\documentclass{acm_proc_article-sp}

\newcommand{\semtitle}{Reputation and Trust on the Semantic Web}
\newcommand{\semkeywords}{Semantic Web, Linked Open Data, Reputation, Security}

\newcommand\MYhyperrefoptions{
       bookmarks=true,bookmarksnumbered=true,
       pdfpagemode={UseOutlines},plainpages=false,pdfpagelabels=true,
       colorlinks=true,linkcolor={black},citecolor={black},urlcolor={black},
       pdftitle={\semtitle},%<!CHANGE!
       pdfsubject={Semantic Web},%<!CHANGE!
       pdfauthor={Thomas A. Grainger},%<!CHANGE!
       pdfkeywords={\semkeywords}
}%<^!CHANGE!

\usepackage[\MYhyperrefoptions,pdftex]{hyperref}
\usepackage[T1]{fontenc}
\usepackage[utf8]{inputenc}
\begin{document}

\title{\semtitle}
\subtitle{}
%
% You need the command \numberofauthors to handle the 'placement
% and alignment' of the authors beneath the title.
%
% For aesthetic reasons, we recommend 'three authors at a time'
% i.e. three 'name/affiliation blocks' be placed beneath the title.
%
% NOTE: You are NOT restricted in how many 'rows' of
% "name/affiliations" may appear. We just ask that you restrict
% the number of 'columns' to three.
%
% Because of the available 'opening page real-estate'
% we ask you to refrain from putting more than six authors
% (two rows with three columns) beneath the article title.
% More than six makes the first-page appear very cluttered indeed.
%
% Use the \alignauthor commands to handle the names
% and affiliations for an 'aesthetic maximum' of six authors.
% Add names, affiliations, addresses for
% the seventh etc. author(s) as the argument for the
% \additionalauthors command.
% These 'additional authors' will be output/set for you
% without further effort on your part as the last section in
% the body of your article BEFORE References or any Appendices.

\numberofauthors{1} %  in this sample file, there are a *total*
% of EIGHT authors. SIX appear on the 'first-page' (for formatting
% reasons) and the remaining two appear in the \additionalauthors section.
%
\author{
% You can go ahead and credit any number of authors here,
% e.g. one 'row of three' or two rows (consisting of one row of three
% and a second row of one, two or three).
%
% The command \alignauthor (no curly braces needed) should
% precede each author name, affiliation/snail-mail address and
% e-mail address. Additionally, tag each line of
% affiliation/address with \affaddr, and tag the
% e-mail address with \email.
%
% 1st. author
\alignauthor
\href{http://graingert.co.uk/foaf.rdf\#me}{Thomas Grainger}\\
       \affaddr{University of Southampton}\\
       \affaddr{University Road}\\
       \affaddr{Southampton, UK}\\
       \email{\protect\href{mailto:t.grainger@ecs.soton.ac.uk}{t.grainger@ecs.soton.ac.uk}}
}
% There's nothing stopping you putting the seventh, eighth, etc.
% author on the opening page (as the 'third row') but we ask,
% for aesthetic reasons that you place these 'additional authors'
% in the \additional authors block, viz.
\date{\today}
% Just remember to make sure that the TOTAL number of authors
% is the number that will appear on the first page PLUS the
% number that will appear in the \additionalauthors section.

\maketitle
\begin{abstract}
Trust is a method of dealing with uncertainty and in Semantic Web where any agent can make an assertion about any object and every one of these statements is treated equally, uncertainty can be rife.  This issue can be solved by using various trust approaches which can be combined during application based on cost/benefit analysis of the particular situation.  This report investigates trust systems from both a Semantic Web and general perspective web technology perspective, with a particular focus on the PGP Web of Trust and the TrustMail Semantic Web system.  After critical analysis of available technology this report concludes that a combination of strategies can be used to provide a more versatile hybrid approach that performs better than either a purely cryptographic or Semantic Web approach.
\end{abstract}

% A category with the (minimum) three required fields
%TODO: \category{H.4}{Information Systems Applications}{Miscellaneous}
%A category including the fourth, optional field follows...
%TODO: \category{D.2.8}{Software Engineering}{Metrics}[complexity measures, performance measures]

%TODO: \terms{Theory}

% Introduction: What is trust, from the various papers. RDF is Subject Predicate Object statements~\cite{rdf}. These are _all_ true in the semantic web graph. We're interested in the sub-graphs we can trust.

% Existing trust systems: PGP, WOT etc Why these require semantics.

% Adding the SW: The various trust systems and algorithms that use the semantic web.

% Ontologies: Comparison of onts and relating them to existing systems. Link between GPG and the golbeck system.

\keywords{\semkeywords} % NOT required for Proceedings

\section{Introduction}
The OED defines `trust' as ``confidence in or reliance on some quality or attribute of a person or thing, or the truth of a statement''.  However, within the context of technology literature this word has taken on a domain specific definition.  O'Hara  et al~\cite{ohara_trust_2004} describe trust as ``a method of dealing with uncertainty'' specifically with regards to the actions an agent may take. Similarly, though more generally,  Golbeck et al~\cite{golbeck_trust_2003} describes trust in terms of users and the statements they make.  Kalfoglou et al note that agents could refer to something as simple as distributed objects or as complex as autonomous entities~\cite{kalfoglou_emergent_2004}.

Trust is an integral issue within Semantic Web as anyone, or more specifically any agent, can make assertions about any object with a valid URI and each of these assertions are considered with equivalent certainty.  While this makes merging datasets trivial within the Semantic Web\cite{golbeck_trust_2003}, this is not a desirable mode of operation for collaboration, automatic discovery or composition of agents and services\footnote{A self-declared unreliable information source: \url{http://data.totl.net/steve.rdf}}. As such a purely Semantic Web approach to this problem is not fit for purpose and could benefit from an amalgamation of trust systems in use before the Semantic Web was developed.  This could lead to an enhanced system of trust which is better able to handle a secure, distributed Semantic Web trust network.

\section{Trust Systems without the Semantic Web}
The PGP Web of Trust (WOT)\cite{_how_1999} splits it's trust model into two distinct attributes, `validity' and `trust'.   Validity is confidence that a particular public key belongs to its purported owner, other users can vouch for the validity of this public key however this requires trusting said people's endorsement - this is where the second metric is required.  Trust is the confidence one has in the judgement of the users who are asserting the validity of the particular public key.  A user's determination of the validity of a particular key can come from a direct determination of validity or via the transitive assertions of trusted users i.e. user A trusts user B who trusts user C then user A transitively trusts user C.

This is similar to the approach taken by Golbeck, in which confidence in the source of a statement does not necessarily relate to the confidence in the validity of the content.

The PGP WOT is implemented using a graph of weighted edges between public keys, where an edge from $k_A$ to $k_B$ is defined as a message referencing $k_B$ with the cryptographic signature generated by the private key of $k_A$. If $k_A$ trusts $k_B$ and $k_B$ trusts $k_B$ then $k_A$ can trust $k_C$.

While this trust operates through public key cryptography, the semantics of the network are extremely important. For example, Bitcoin is syntactically similar to the PGP WOT - a graph of signatures with signed messages representing weighted edges, however the semantics of the system is significantly different~\cite{nakamoto_bitcoin:_2008}. Within the Bitcoin protocol, each node of the graph (a public key) is limited to having the same total weight of outgoing edges as incoming edges so that the system can represent value transfer instead of trust this is in comparison to PGP WOT in which users are free to make whatever statement, regarding trust connections, they wish.

\section{Semantic Web Trust Systems}
O'Hara et al~\cite{ohara_trust_2004} define five basic trust derivation strategies based on optimism, pessimism, centralisation, investigation and transitivity.  Optimistic trust derivation assumes that all agents are trustworthy until proven otherwise, pessimism is the reflection of this; requiring proof of trustworthiness as all agents are considered untrustworthy until proven otherwise an example of this is the first stage of the PGP web of trust: the manual validation of individual users.  Centralised trust systems rely on a single institution defining the trust of other agents and example of this the WebTrust system~\cite{golbeck_trust_2003} or x.509 certificate authorities (CAs).  Investigation as a trust derivation strategy involves a Bayesian network based trust model, in which agents gradually reveal information to each other until trust is established.  Finally transitivity which involves and requires a network of agents describing mutual trust, this is similar to the social network trust methods used within the PGP WOT.

These strategies are useful within particular situations and confer specific sets of advantages and disadvantages and can, to an extent, be combined e.g. a centralized service which utilises a transitive trust algorithm.  An example of this would be Golbeck's TrustMail service which utilises a centralized trust system that determines a transitive trust value from a graph of social network data which can then be queried by independent agents.  The algorithm below, as described by Golbeck and used by the TrustMail service, determines a calculated trust $t$ from node $i$ to node $n$ using equation~\ref{eq:golbeck}:

\begin{equation}\label{eq:golbeck}
t_{is}=\frac{\sum_{j=0}^{n}{\begin{cases}
(t_{js}*t_{ij}) & \text{ if } t_{ij} \geq t_{js} \\ 
(t_{ij}^2) & \text{ if } x= t_{ij} < t_{js}
\end{cases}}}{\sum_{j=0}^{n}{t_{ij}}}
\end{equation}

where $i$ has $n$ neighbours with paths to $s$.

As stated before, these strategies are chosen based on various cost definitions.  According to the O'Hara~\cite{ohara_trust_2004} definition, the principle function of trust strategies is to provide a way of operating under uncertainty.   The choice of trust strategy depends on the cost of failure (could lead to a catastrophe) and risks.  Costs of failure can include: operational costs, opportunity costs, deficiency costs and Service charges.   Thus a Risk/Cost/Benefit analysis needs to be performed in order to find an optimum strategy.

Reagle states that ``highly interrelated values are trustable'', an example given is that if an agent wishes to validate another agent's key that agent can query the Semantic Web for data relating the key to the agent.  If enough disparate systems declare a particular statement it can be assumed to be true~\cite{reagle_key_2002}. Crucially, it would be possible for a particularly determined adversary to dissemate sufficient false information using multiple services such that legitimate information would be outweighed allowing the true value to be concealed i.e. Sybil\cite{douceur_sybil_2002} attacks.  This has been shown to work in practise using a technique known as ``Google-bombing'' in which several high profile politicians including President Bush, Senator Clinton and Senator Rick Santorum were linked with ``various unprintable phrases''\cite{tom_mcnichol_your_2004}.

Unfortunately, Reagle's report misses a security consideration - that an agent could control the network link to the datasets on the web. This would be a problem where the information pertinent to the agent controlling the link is used in a query of the Semantic Web.   An example consequence of this would be that querying quality rankings of home internet service providers (ISP) would not be necessarily as trustworthy if these rankings were accessed via an ISP that exists within that ranking.

\section{A Comparison of Vocabularies}
Golbeck~\cite{golbeck_trust_2003} notes that hyperlinked pages could be used as a trust network, and agrees that Sematic Web technologies are better equipped than bare hyperlinks to handle trust because the Semantic Web allows concept relationships to be marked up without need of heuristics.

In social web, an Erdos or Bacon number can be used to determine potential trust between friends however, users of FOAF, the Friend of a Friend ontology\cite{dan_brickley_foaf_????}, are unlikely to assume that the declaration of knowledge will also suggest some level of trust. As such, Golbeck published a concrete ontology derived from concepts in the paper~\cite{jennifer_golbeck_trust_2013}. The N-Ary relation ontology design pattern allows a person to trust another person (``trustsPerson'') on a particular subject (``trustsOnSubject'').

The Web of Trust (wot) ontology can be used to describe messages from the PGP network\cite{_web_2004} using concepts such as RDF descriptions of public keys and signed assurances.  Ideally, an ontology could be produced to link the PGP WOT graph with the Trust Ontology and wot ontologies, thus allowing the PGP web of trust to be re-searialized as RDF and reasoned over using the existing trust algorithms. This would allow services such as TrustMail to take advantage of existing data in the PGP WOT and allow services such as GPGMail to take advantage of advanced trust strategies discussed earlier.

A potential basis for this ontology exists in the ``Trust Service Ontology'' described  by Sherchan et al~\cite{sherchan_trust_2010} for semantically marking up . The Trust Service Ontology  allows trust composition and trust propagation as well as semantically describing different types of trust and the relationship between trust concepts and algorithms.

\section{Conclusions}
%\end{document}  % This is where a 'short' article might terminate

Neither a purely Semantic Web or cryptographic solution to the problem of trust as useful as a combination of systems.  Both approaches could be enhanced by the security or algorithmic developments of the other.  Future directions for research and development could focus on producing an RDF serialization of the PGP WOT graph using various Semantic Web ontologies and trust approaches to build a more versatile system allowing agents to take advantage of the PGP WOT graph as part of their automatic discovery as well as composition of services and other agents.

%ACKNOWLEDGMENTS are optional
\section{Acknowledgments}
The author would like to thank Dr Nicholas Gibbins for suggesting a title and preliminary reading for this report.

%
% The following two commands are all you need in the
% initial runs of your .tex file to
% produce the bibliography for the citations in your paper.
\bibliographystyle{abbrv}
\bibliography{i-d,rfc,sigproc,semweb}  % sigproc.bib is the name of the Bibliography in this case

% You must have a proper ".bib" file
%  and remember to run:
% latex bibtex latex latex
% to resolve all references
%
% ACM needs 'a single self-contained file'!
%
% That's all folks!
\end{document}
