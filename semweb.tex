\documentclass{acm_proc_article-sp}

\newcommand{\semtitle}{Reputation and Trust on the Semantic Web}
\newcommand{\semkeywords}{Semantic Web, Linked Open Data, Reputation, Security}

\newcommand\MYhyperrefoptions{
       bookmarks=true,bookmarksnumbered=true,
       pdfpagemode={UseOutlines},plainpages=false,pdfpagelabels=true,
       colorlinks=true,linkcolor={black},citecolor={black},urlcolor={black},
       pdftitle={\semtitle},%<!CHANGE!
       pdfsubject={Semantic Web},%<!CHANGE!
       pdfauthor={Thomas A. Grainger},%<!CHANGE!
       pdfkeywords={\semkeywords}
}%<^!CHANGE!

\usepackage[\MYhyperrefoptions,pdftex]{hyperref}

\begin{document}

\title{\semtitle}
\subtitle{}
%
% You need the command \numberofauthors to handle the 'placement
% and alignment' of the authors beneath the title.
%
% For aesthetic reasons, we recommend 'three authors at a time'
% i.e. three 'name/affiliation blocks' be placed beneath the title.
%
% NOTE: You are NOT restricted in how many 'rows' of
% "name/affiliations" may appear. We just ask that you restrict
% the number of 'columns' to three.
%
% Because of the available 'opening page real-estate'
% we ask you to refrain from putting more than six authors
% (two rows with three columns) beneath the article title.
% More than six makes the first-page appear very cluttered indeed.
%
% Use the \alignauthor commands to handle the names
% and affiliations for an 'aesthetic maximum' of six authors.
% Add names, affiliations, addresses for
% the seventh etc. author(s) as the argument for the
% \additionalauthors command.
% These 'additional authors' will be output/set for you
% without further effort on your part as the last section in
% the body of your article BEFORE References or any Appendices.

\numberofauthors{1} %  in this sample file, there are a *total*
% of EIGHT authors. SIX appear on the 'first-page' (for formatting
% reasons) and the remaining two appear in the \additionalauthors section.
%
\author{
% You can go ahead and credit any number of authors here,
% e.g. one 'row of three' or two rows (consisting of one row of three
% and a second row of one, two or three).
%
% The command \alignauthor (no curly braces needed) should
% precede each author name, affiliation/snail-mail address and
% e-mail address. Additionally, tag each line of
% affiliation/address with \affaddr, and tag the
% e-mail address with \email.
%
% 1st. author
\alignauthor
Thomas Grainger\\
       \affaddr{University of Southampton}\\
       \affaddr{University Road}\\
       \affaddr{Southampton, UK}\\
       \email{tag1g09@ecs.soton.ac.uk}
}
% There's nothing stopping you putting the seventh, eighth, etc.
% author on the opening page (as the 'third row') but we ask,
% for aesthetic reasons that you place these 'additional authors'
% in the \additional authors block, viz.
\date{\today}
% Just remember to make sure that the TOTAL number of authors
% is the number that will appear on the first page PLUS the
% number that will appear in the \additionalauthors section.

\maketitle
\begin{abstract}
TODO: Abstract
\end{abstract}

% A category with the (minimum) three required fields
%TODO: \category{H.4}{Information Systems Applications}{Miscellaneous}
%A category including the fourth, optional field follows...
%TODO: \category{D.2.8}{Software Engineering}{Metrics}[complexity measures, performance measures]

%TODO: \terms{Theory}

\keywords{\semkeywords} % NOT required for Proceedings

\section{Introduction}


\section{Existing Trust Systems}
GPG WOT
Bitcoin

\section{Adding Semantic Web}
Semantic Web Trust Ontologies/ Network types
WOT/RDF
Key Free Trust

\section{Conclusions and Future Work}
We worked hard, but discovered very little.
%\end{document}  % This is where a 'short' article might terminate

%ACKNOWLEDGMENTS are optional
\section{Acknowledgments}
The author would like to thank Dr Nicholas Gibbins for suggesting a title and preliminary reading for this report.

%
% The following two commands are all you need in the
% initial runs of your .tex file to
% produce the bibliography for the citations in your paper.
\bibliographystyle{abbrv}
\bibliography{i-d,rfc,sigproc,main}  % sigproc.bib is the name of the Bibliography in this case
% You must have a proper ".bib" file
%  and remember to run:
% latex bibtex latex latex
% to resolve all references
%
% ACM needs 'a single self-contained file'!
%
% That's all folks!
\end{document}
