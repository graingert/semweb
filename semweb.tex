\documentclass{acm_proc_article-sp}

\newcommand{\semtitle}{Reputation and Trust on the Semantic Web}
\newcommand{\semkeywords}{Semantic Web, Linked Open Data, Reputation, Security}

\newcommand\MYhyperrefoptions{
       bookmarks=true,bookmarksnumbered=true,
       pdfpagemode={UseOutlines},plainpages=false,pdfpagelabels=true,
       colorlinks=true,linkcolor={black},citecolor={black},urlcolor={black},
       pdftitle={\semtitle},%<!CHANGE!
       pdfsubject={Semantic Web},%<!CHANGE!
       pdfauthor={Thomas A. Grainger},%<!CHANGE!
       pdfkeywords={\semkeywords}
}%<^!CHANGE!

\usepackage[\MYhyperrefoptions,pdftex]{hyperref}

\begin{document}

\title{\semtitle}
\subtitle{}
%
% You need the command \numberofauthors to handle the 'placement
% and alignment' of the authors beneath the title.
%
% For aesthetic reasons, we recommend 'three authors at a time'
% i.e. three 'name/affiliation blocks' be placed beneath the title.
%
% NOTE: You are NOT restricted in how many 'rows' of
% "name/affiliations" may appear. We just ask that you restrict
% the number of 'columns' to three.
%
% Because of the available 'opening page real-estate'
% we ask you to refrain from putting more than six authors
% (two rows with three columns) beneath the article title.
% More than six makes the first-page appear very cluttered indeed.
%
% Use the \alignauthor commands to handle the names
% and affiliations for an 'aesthetic maximum' of six authors.
% Add names, affiliations, addresses for
% the seventh etc. author(s) as the argument for the
% \additionalauthors command.
% These 'additional authors' will be output/set for you
% without further effort on your part as the last section in
% the body of your article BEFORE References or any Appendices.

\numberofauthors{1} %  in this sample file, there are a *total*
% of EIGHT authors. SIX appear on the 'first-page' (for formatting
% reasons) and the remaining two appear in the \additionalauthors section.
%
\author{
% You can go ahead and credit any number of authors here,
% e.g. one 'row of three' or two rows (consisting of one row of three
% and a second row of one, two or three).
%
% The command \alignauthor (no curly braces needed) should
% precede each author name, affiliation/snail-mail address and
% e-mail address. Additionally, tag each line of
% affiliation/address with \affaddr, and tag the
% e-mail address with \email.
%
% 1st. author
\alignauthor
Thomas Grainger\\
       \affaddr{University of Southampton}\\
       \affaddr{University Road}\\
       \affaddr{Southampton, UK}\\
       \email{tag1g09@ecs.soton.ac.uk}
}
% There's nothing stopping you putting the seventh, eighth, etc.
% author on the opening page (as the 'third row') but we ask,
% for aesthetic reasons that you place these 'additional authors'
% in the \additional authors block, viz.
\date{\today}
% Just remember to make sure that the TOTAL number of authors
% is the number that will appear on the first page PLUS the
% number that will appear in the \additionalauthors section.

\maketitle
\begin{abstract}
TODO: Abstract
\end{abstract}

% A category with the (minimum) three required fields
%TODO: \category{H.4}{Information Systems Applications}{Miscellaneous}
%A category including the fourth, optional field follows...
%TODO: \category{D.2.8}{Software Engineering}{Metrics}[complexity measures, performance measures]

%TODO: \terms{Theory}

% Introduction: What is trust, from the various papers. RDF is Subject Predicate Object statements~\cite{rdf}. These are _all_ true in the semantic web graph. We're intersted in the subgraphs we can trust.

% Existing trust systems: PGP, WOT etc Why these require semantics.

% Adding the SW: The various trust systems and algorithms that use the semantic web.

% Ontologies: Comparison of onts and relating them to existing systems.

\keywords{\semkeywords} % NOT required for Proceedings

\section{Introduction}
The OED defines `trust' as ``confidence in or reliance on some quality or attribute of a person or thing, or the truth of a statement''.  However, within the context of technology literature this word has taken on a domain specific definition.  O'Hara  et al~\cite{ohara_trust_2004} describe trust as ``a method of dealing  with uncertainty'' specifically with regards to the actions an agent may take. Similarly, though more generally,  Golbeck et al~\cite{golbeck_trust_2003} describes trust  in terms of users and the statements they make.

Trust is an integral issue within Semantic Web as anyone, or more specifically any agent, can make assertions about any object with a valid URI and each of these assertions are considered with equivalent certainty.  While this makes merging datasets trivial within the Semantic Web\cite{golbeck_trust_2003}, this is not a desirable mode of operation for collaboration, automatic discovery or composition of agents and services\footnote{A self-declared unreliable information source: \url{http://data.totl.net/steve.rdf}}.


TODO: THUS.... general point you are making in the paper.

\section{Trust Systems without the Semantic Web}
The PGP Web of Trust (WOT)\cite{_how_pgp_1999} splits it's trust model into two distinct attributes, `validity' and `trust'.   Validity is confidence that a particular public key belongs to its purported owner, other users can vouch for the validity of this public key however this requires trusting said people's endorsement - this is where the second metric is required.  Trust is the confidence one has in the judgement of the users who are asserting the validity of the particular public key.  A user's determination of the validity of a particular key can come from a direct determination of validity or via the transitive assertions of trusted users i.e. user A trusts user B who trusts user C then user A transitively trusts user C.

The PGP WOT is implemented using a graph of weighted edges between public keys, where an edge from $k_A$ to $k_B$ is defined as a message referencing $k_B$ with the cryptographic signature generated by the private key of $k_A$. If $k_A$ trusts $k_B$ and $k_B$ trusts $k_B$ then $k_A$ can trust $k_C$.

While this trust operates through public key cryptography, the semantics of the network are extremely important.  For example, Bitcoin is syntactically similar to the PGP WOT - a graph of signatures with signed messages representing weighted edges, however the semantics of the system is significantly different\cite{bitcoin}.  Within the Bitcoin protocol, each node of the graph (a public key) is limited to having the same total weight of outgoing edges as incoming edges so that the system can represent value transfer instead of trust this is in comparison to PGP WOT in which users are free to make whatever statement, regarding trust connections, they wish.

\section{Adding Semantic Web}
As ``the heart of the Semantic Web'' O'Hara et al~\cite{ohara_trust_2004} define five basic trust derivation strategies: Optimism, Pessimism, Centralized, Investigation, Transitivity.

\paragraph{Optimism} assumes that agents are trustworthy ``unless proven otherwise''
\paragraph{Pessimism} is the opposite of Optimism, assuming that agents are untrustworthy.
\paragraph{Centralized} trust systems use a single institution to define the trust of other agents
\paragraph{Investigation} trust is allocated after the investigation of other agents. Attributes of those agents that reduce uncertainty are explicitly investigated.
\paragraph{Transitivity} requires a network of agents describing mutual trust. Similar to social network trust methods used with the PGP WOT and The Trust Project~\cite{_how_pgp_1999,golbeck_trust_2003}.

These strategies are chosen based on various cost definitions ``The main point of a trust strategy is to provide a way to operate under uncertainty'': Operational costs, Opportunity cost, Deficiency cost, Service charges. The choice of trust strategy depends on the cost of failure (could lead to a catastrophe) and risks.  Risk/Cost/Benifit analysis.

Independent agents and providers of knowledge. These same trust trust algorithms can be described semantically for service oriented systems using the ``Trust Service Ontology'' described by Sherchan et al~\cite{sherchan_trust_2010}. The Trust Service Ontology allows trust composition and trust propagation and semantically describes different types of trust: ``proposed approach differentiates and explicitly
defines the relationship between trust concepts and ... trust algorithms''

The classes of trust defined are: Bootstrapped Trust, GlobalTrust, PersonalisedTrust, DirectTrust, PropagtedTrust and CompositeTrust.

\subsection{A Comparison of Vocabularies}
WOT/RDF
Key Free Trust

``For example, just because a person can confirm the source of documents does not license them to trust the content of those documents. Indeed, confirmation that some documents are the work of a noted forger gives strong support to distrusting their content. This project addresses ``trust'' as credibility or reliability in a much more human''

Confidence in the source of a statement does not necessarily 


What is an agent? Simple distributed objects, complex autonomous entities.
\cite{kalfoglou_emergent_2004}

About users
Confidence of the author is important
Social network analysis
``TRELLIS is an interactive tool that helps users annotate the
rationale for their decisions, hypotheses, and opinions as they analyze information from various sources''\cite{gil_trusting_2002}

Hyperlinked pages could be used as a trust network, SW is better because the SW allows concept relationships to be marked up, no need to use heuristics
In social web, Erdos or Bacon number can be used to determine potential trust between friends.
foaf:knows should not be used to give trust, as that's semantically incorrect
defines a trust ontology with 9 levels of trust.

N-Ary relationship pattern to add ``TrustsRe'' links, can trust a person ``trustsPerson'' on a particular subject ``trustsOnSubject''
The calculated trust $t$ from node $i$ to node $n$ is given by the following function:
\begin{equation}
t_{is}=\frac{\sum_{j=0}^{n}{\begin{cases}
(t_{js}*t_{ij}) & \text{ if } t_{ij} \geq t_{js} \\ 
(t_{ij}^2) & \text{ if } x= t_{ij} < t_{js}
\end{cases}}}{\sum_{j=0}^{n}{t_{ij}}}
\end{equation}
where $i$ has $n$ neighbors with paths to $s$.

centralized web service to maintain trust data
A centralized repository that uses transitive algorithms.
TrustMail queries web service 
\cite{golbeck_trust_2003}

A concrete ontology for trust, with the same intention as discussed in the paper.
\cite{jennifer_golbeck_trust_2013}


``The degree to which an agent considers an assertion to be true for a given context. While the term ``trust'' is often used to denote a very high degree of confidence, there is an associated risk of the assertions being wrong.''

Public Key Infrastructure PKI using certificate hierarchy.
Needs a chain of signing to the root CA.

Web of Trust popularized by PGP, similar to Bacon numbers from \cite{golbeck_trust_2003}

``The pervasive use of digest values to identify the statements in the Semantic Web will engender a preponderance of evidence for trust without cryptography.''

Declares that highly inter-related values are trust-able, does not note however that it is vulnerable to Sybil\cite{douceur_sybil_2002}
\cite{reagle_key_2002}

Web of trust ontology can be used to serialize the PGP WOT
\cite{_web_2004}

\section{Conclusions and Future Work}
We worked hard, but discovered very little.
%\end{document}  % This is where a 'short' article might terminate

%ACKNOWLEDGMENTS are optional
\section{Acknowledgments}
The author would like to thank Dr Nicholas Gibbins for suggesting a title and preliminary reading for this report.

%
% The following two commands are all you need in the
% initial runs of your .tex file to
% produce the bibliography for the citations in your paper.
\bibliographystyle{abbrv}
\bibliography{i-d,rfc,sigproc,semweb}  % sigproc.bib is the name of the Bibliography in this case

% You must have a proper ".bib" file
%  and remember to run:
% latex bibtex latex latex
% to resolve all references
%
% ACM needs 'a single self-contained file'!
%
% That's all folks!
\end{document}
